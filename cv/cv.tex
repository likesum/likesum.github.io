%%%%%%%%%%%%%%%%%%%%%%%%%%%%%%%%%%%%%%%%%
% Medium Length Professional CV
% LaTeX Template
% Version 2.0 (8/5/13)
%
% This template has been downloaded from:
% http://www.LaTeXTemplates.com
%
% Original author:
% Trey Hunner (http://www.treyhunner.com/)
%
% Important note:
% This template requires the resume.cls file to be in the same directory as the
% .tex file. The resume.cls file provides the resume style used for structuring the
% document.
%
%%%%%%%%%%%%%%%%%%%%%%%%%%%%%%%%%%%%%%%%%

%----------------------------------------------------------------------------------------
%   PACKAGES AND OTHER DOCUMENT CONFIGURATIONS
%----------------------------------------------------------------------------------------

\documentclass{resume} % Use the custom resume.cls style

\usepackage[left=0.75in,top=0.6in,right=0.75in,bottom=0.6in]{geometry} % Document margins
\usepackage{hyperref}
\urlstyle{rm}
\newcommand{\B}[1]{\textbf{#1}}
\newcommand{\pub}[1]{\textrm{\textit{#1}}}

\name{Zhihao Xia} % Your name
% \address{Computer Science and Engineering Department, Washington University in St. Louis} % Your address
\address{\href{mailto:zxia@adobe.com}{zxia@adobe.com} \\ \url{https://likesum.github.io}} % Your phone number and email

\begin{document}

%----------------------------------------------------------------------------------------
%   RESEARCH INTERESTS SECTION
%----------------------------------------------------------------------------------------

\begin{rSection}{Research interests}
My research interests include computer vision, computational photography, and machine learning. I am particularly interested in building methods that can recover different aspects of visual appearance (geometry, light, colors, etc.) from images, by using both physics-based reasoning and machine learning.

\end{rSection}

%----------------------------------------------------------------------------------------
%   EDUCATION SECTION
%----------------------------------------------------------------------------------------

\begin{rSection}{Education}

{\bf Washington University in St. Louis} \hfill {\em Sep 2017 - Jun 2021} \\
Ph.D. in Computer Science \& Engineering~~~~~~~~Advisor: Ayan Chakrabarti \\

{\bf University of Science and Technology of China} \hfill {\em Sep 2013 - Jun 2017} \\
B.S. in Computer Science, School of the Gifted Young \\
% Overall GPA: 3.52/4.3 \\
Honor: National Scholarship (Top 2\%)

\end{rSection}

%----------------------------------------------------------------------------------------
%   WORK EXPERIENCE SECTION
%----------------------------------------------------------------------------------------
\begin{rSection}{Experience}

%------------------------------------------------

\begin{rSubsection}{Adobe}{Jun 2021 - Present}{Research Scientist II}{CA, USA}
\item Using computational photography and machine leanring to build next-gen camera for mobile devices.

\end{rSubsection}


%------------------------------------------------

\begin{rSubsection}{Google Research}{May 2020 - Dec 2020}{Research Intern}{WA, USA}
\item Worked with Supreeth Achar and Jason Lawrence on face relighting and normal estimation under challenging visible light environments by supplementing a near-infared image.

\end{rSubsection}

%------------------------------------------------

\begin{rSubsection}{Adobe Research}{May 2019 - Aug 2019}{Research Intern}{CA, USA}
\item Worked with Federico Perazzi, Micha\"el Gharbi, Kalyan Sunkavalli on Basis Prediction Networks.
\item Research was transferred into products and published in CVPR 2020.

\end{rSubsection}

%------------------------------------------------
\begin{rSubsection}{Washington University in St. Louis}{Sep 2017 - Jun 2021}{Research Assistant}{MO, USA}
\item Member of WashU vision and learning group directed by Prof. Ayan Chakrabarti.
\item Computer Vision, Computational Photography and Deep Learning.
\end{rSubsection}

%------------------------------------------------

\begin{rSubsection}{King Abdullah University of Science and Technology }{Jan 2017 - May 2017}{Visiting Student Researcher}{Jeddah, SAU}
\item Worked with Prof. Xin Gao on recognition of polyadenylation signal (PAS) in human and mouse genes with CNN.
\end{rSubsection}

%------------------------------------------------

\begin{rSubsection}{National University of Singapore}{July 2016 - Nov 2016}{Visiting Student Researcher}{Singapore}
\item Worked with Prof. Richard Ma on investigating Internet topology with Multiple Hidden Markov Chains.
\end{rSubsection}

%------------------------------------------------

\begin{rSubsection}{University of Science and Technology of China}{June 2015 - June 2016}{Undergraduate Research Assistant}{Hefei, China}
\item Topic diffusion analytics in social media based on machine learning.
\end{rSubsection}

\end{rSection}

\clearpage
%----------------------------------------------------------------------------------------

%----------------------------------------------------------------------------------------
%   PUBLICATION SECTION
%----------------------------------------------------------------------------------------
\sectionskip
\MakeUppercase{\bf Publication} \href{https://scholar.google.com/citations?hl=en&user=Rc4ZMCEAAAAJ}{[Google Scholar]}
\sectionlineskip
\hrule % Horizontal line

% \B{Refereed}
\begin{enumerate}
  \item Zheng Ding, Xuaner (Cecilia) Zhang, \B{Zhihao Xia}, Lars Jebe, Zhuowen Tu, Xiuming Zhang. ``DiffusionRig: Learning Personalized Priors for Facial Appearance Editing'', \pub{CVPR}, 2023.
  \item Ke Wang, Michaël Gharbi, He Zhang, \B{Zhihao Xia}, Eli Shechtman. ``Semi-supervised Parametric Real-world Image Harmonization'', \pub{CVPR}, 2023.
  \item Ethan Tseng, Yuxuan Zhang, Lars Jebe, Xuaner (Cecilia) Zhang, \B{Zhihao Xia}, Yifei Fan, Felix Heide*, Jiawen Chen* ``Neural Photo-Finishing'', \pub{SIGGRAPH Asia}, 2022.
  \item Ilya Chugunov, Yuxuan Zhang, \B{Zhihao Xia}, Xuaner (Cecilia) Zhang, Jiawen Chen, Felix Heide. ``The Implicit Values of A Good Hand Shake: Handheld Multi-Frame Neural Depth Refinement'', \pub{CVPR}, 2022 \B{(Oral)}.
  \item \B{Zhihao Xia}, Jason Lawrence, Supreeth Achar. ``A Dark Flash Normal Camera'', \pub{ICCV}, 2021.
  \item \B{Zhihao Xia}, Micha\"el Gharbi, Federico Perazzi, Kalyan Sunkavalli, Ayan Chakrabarti. ``Deep Denoising of Flash and No-Flash Pairs for Photography in Low-Light Environments'', \pub{CVPR}, 2021.
  \item \B{Zhihao Xia}, Federico Perazzi, Micha\"el Gharbi, Kalyan Sunkavalli, Ayan Chakrabarti. ``Basis Prediction Networks for Effective Burst Denoising with Large Kernels'', \pub{CVPR}, 2020.
  \item \B{Zhihao Xia}, Patrick Sullivan, Ayan Chakrabarti. ``Generating and Exploiting Probabilistic Monocular Depth Estimates'', \pub{CVPR}, 2020 \B{(Oral)}.
  \item \B{Zhihao Xia} and Ayan Chakrabarti. ``Identifying Recurring Patterns with Deep Neural Networks for Natural Image Denoising'', \pub{WACV}, 2020.
  \item \B{Zhihao Xia} and Ayan Chakrabarti. ``Training Image Estimators without Image Ground-Truth'', \pub{NeurIPS}, 2019 \B{(Spotlight)}
  \item \B{Zhihao Xia}, Yu Li, Bin Zhang, Yuhui Hu, Wei Chen and Xin Gao. ``DeeReCT-PolyA: a robust and generic deep learning method for PAS identification'', \pub{Bioinformatics}, 2018.
  \item Yu Sun, \B{Zhihao Xia} and Ulugbek S. Kamilov. ``Efficient and accurate inversion of multiple scattering with deep learning'', \pub{Optics express} 26(11): 14678-14688, 2018.
  \end{enumerate}

  % \B{Preprints}
  % \begin{enumerate}
  % \setcounter{enumi}{6}
  % \end{enumerate}

  %----------------------------------------------------------------------------------------
  %   TECHNICAL STRENGTHS SECTION
  %----------------------------------------------------------------------------------------

  \begin{rSection}{Skills}
  % \vspace{-1em}
  Python: Numpy, Tensorflow, PyTorch. C/C++. Matlab
  \end{rSection}

  \sectionskip
  \MakeUppercase{\bf Service} % Section title
  \sectionlineskip
  \hrule % Horizontal line

  \begin{itemize}
  \item Reviewer for IJCV, TIP, CVPR (2021), ICCV (2021, 2019), IJCAI (2021, SPC), WACV (2020)\\
  \vspace{-1em}
  \item Assistant to Instructor, WashU Fall 2018 CSE 559A: Computer Vision. Developed and held recitation lectures, assisted with grading, staffed office hours to assist students\\
  \vspace{-1em}
  \item Co-organizor: WashU CS Graphics, Vision, and Imaging Seminar, Fall 2019 and Spring 2020
  \end{itemize}

  % \begin{rSection}{References}

  % Available on request.

  % \end{rSection}

  %----------------------------------------------------------------------------------------
  %   EXAMPLE SECTION
  %----------------------------------------------------------------------------------------

  %\begin{rSection}{Section Name}

  %Section content\ldots

  %\end{rSection}

  %----------------------------------------------------------------------------------------

  \end{document}
